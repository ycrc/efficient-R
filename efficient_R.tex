\documentclass[ignorenonframetext,]{beamer}
\setbeamertemplate{caption}[numbered]
\setbeamertemplate{caption label separator}{: }
\setbeamercolor{caption name}{fg=normal text.fg}
\beamertemplatenavigationsymbolsempty
\usepackage{lmodern}
\usepackage{amssymb,amsmath}
\usepackage{ifxetex,ifluatex}
\usepackage{fixltx2e} % provides \textsubscript
\ifnum 0\ifxetex 1\fi\ifluatex 1\fi=0 % if pdftex
  \usepackage[T1]{fontenc}
  \usepackage[utf8]{inputenc}
\else % if luatex or xelatex
  \ifxetex
    \usepackage{mathspec}
  \else
    \usepackage{fontspec}
  \fi
  \defaultfontfeatures{Ligatures=TeX,Scale=MatchLowercase}
\fi
% use upquote if available, for straight quotes in verbatim environments
\IfFileExists{upquote.sty}{\usepackage{upquote}}{}
% use microtype if available
\IfFileExists{microtype.sty}{%
\usepackage{microtype}
\UseMicrotypeSet[protrusion]{basicmath} % disable protrusion for tt fonts
}{}
\newif\ifbibliography
\hypersetup{
            pdftitle={Writing Efficient R Code},
            pdfauthor={Ben Evans; Stephen Weston},
            pdfborder={0 0 0},
            breaklinks=true}
\urlstyle{same}  % don't use monospace font for urls

% Prevent slide breaks in the middle of a paragraph:
\widowpenalties 1 10000
\raggedbottom

\AtBeginPart{
  \let\insertpartnumber\relax
  \let\partname\relax
  \frame{\partpage}
}
\AtBeginSection{
  \ifbibliography
  \else
    \let\insertsectionnumber\relax
    \let\sectionname\relax
    \frame{\sectionpage}
  \fi
}
\AtBeginSubsection{
  \let\insertsubsectionnumber\relax
  \let\subsectionname\relax
  \frame{\subsectionpage}
}

\setlength{\parindent}{0pt}
\setlength{\parskip}{6pt plus 2pt minus 1pt}
\setlength{\emergencystretch}{3em}  % prevent overfull lines
\providecommand{\tightlist}{%
  \setlength{\itemsep}{0pt}\setlength{\parskip}{0pt}}
\setcounter{secnumdepth}{0}

\title{Writing Efficient R Code}
\author{Ben Evans \and Stephen Weston}
\institute{YCRC}
\date{August 22, 2018}

\begin{document}
\frame{\titlepage}

\begin{frame}{Outline \textbar{} What we'll cover today}

\begin{itemize}
\tightlist
\item
  R performance tips: patterns to use and avoid
\item
  Loops and vectors, byte-compilation
\item
  Profiling and Benchmarking
\item
  Memory management, large tables
\end{itemize}

\end{frame}

\begin{frame}{General advice}

\begin{itemize}
\tightlist
\item
  If you don't understand something, try some experiments
\item
  Browse the documentation, learn its jargon
\item
  Break your code into functions when appropriate
\item
  Use functions to reduce the need for global variables
\item
  Write tests for your functions
\item
  Use git to keep track of changes
\end{itemize}

\end{frame}

\begin{frame}{Is R slow?}

R programs can be slow, but well written R programs are usually fast
enough.

\begin{itemize}
\tightlist
\item
  Designed to make programming easier
\item
  Speed was not the primary design criteria
\item
  Slow programs often a result of bad programming practises or not
  understanding how R works
\item
  There are various options for calling C or C++ functions from R
\end{itemize}

\end{frame}

\begin{frame}[fragile]{Code tuning advice \textbar{} Premature
optimization is the root of all evil -- Donald Knuth}

Tuning code is tricky and not intuitive, so be methodical.

\begin{itemize}
\tightlist
\item
  Don't get carried away with micro-optimizations
\item
  Pre-allocate result vectors, be careful to avoid duplication of
  objects
\item
  Profile your code and run benchmarks
\item
  Byte-compile with \textit{cmpfun}, or call a compiled language
  (e.g.~C, C++)
\item
  Become familiar with R's vector functions and apply functions
\item
  Learn to use a parallel computing package
\item
  Consider specialized packages: \texttt{data.table},
  \texttt{bigmemory}, \texttt{plyr}, \texttt{RSQLite}
\item
  Don't use an R GUI when performance is important
\item
  Use monitoring tools such as \texttt{textit}
\end{itemize}

\end{frame}

\begin{frame}{Resources}

\begin{itemize}
\tightlist
\item
  R Book: \url{http://adv-r.had.co.nz}
\item
  R Manuals: \url{http://cran.r-project.org/manuals.html}
\item
  R Inferno:
  \href{\%5Bhttp://www.burns-stat.com/documents/books/the-r-inferno\%5D}{http://www.burns-stat.com/documents/books/the-r-inferno}
\end{itemize}

\end{frame}

\end{document}
